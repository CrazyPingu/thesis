%
%%%%%%%%%%%%%%%%%%%%%%%%%%%%%%%%%%%%%%%%
\pagenumbering{roman}                   %serve per mettere i numeri romani
\chapter*{Introduzione}                 %crea l'introduzione (un capitolo
                                        %   non numerato)
%%%%%%%%%%%%%%%%%%%%%%%%%%%%%%%%%%%%%%%%%imposta l'intestazione di pagina
\rhead[\fancyplain{}{\bfseries
INTRODUZIONE}]{\fancyplain{}{\bfseries\thepage}}
\lhead[\fancyplain{}{\bfseries\thepage}]{\fancyplain{}{\bfseries
INTRODUZIONE}}
%%%%%%%%%%%%%%%%%%%%%%%%%%%%%%%%%%%%%%%%%aggiunge la voce Introduzione
                                        %   nell'indice
% \addcontentsline{toc}{chapter}{Introduzione}
% Negli ultimi anni, l'adozione accelerata delle tecnologie avanzate nell'ambito dell'Industria 4.0 ha portato a rivoluzionare molteplici settori, tra cui il turismo~\cite{Turismo} e le attività escursionistiche. Questo cambiamento ha introdotto una serie di tecnologie innovative tra cui spiccano il Digital Twin, il Digital Shadow e il Digital Thread.

% Il contesto in cui si colloca questa tesi è proprio questo scenario di trasformazione. In particolare, questa tesi propone una piattaforma web progettata per offrire un servizio agli amanti delle escursioni nella regione Emilia-Romagna. L'obiettivo principale di questa piattaforma è migliorare l'esperienza degli escursionisti, integrando in modo intelligente i concetti chiave di Digital Twin, Digital Shadow e Digital Thread.

% La piattaforma consente agli appassionati di escursioni non solo di esplorare e pianificare i loro percorsi, rendendoli disponibili sotto forma di mappe interattive, ma offre anche la possibilità di contribuire attivamente. Gli utenti possono arricchire la piattaforma aggiungendo informazioni utili sui singoli punti di interesse lungo i percorsi, come ad esempio gli orari di apertura dei musei o le descrizioni dettagliate di siti storici, migliorando così l'esperienza per tutti gli utenti. Gli amministratori della piattaforma, nel frattempo, hanno la possibilità di mantenere un certo livello di controllo sui dati inseriti, garantendo l'accuratezza e la qualità delle informazioni condivise.

% \newpage
% Riguardo alla struttura di questa tesi, il documento è organizzato in tre sezioni principali:
% \begin{enumerate}
%     \item Contesto \\
%     Questa sezione offre un'analisi approfondita dei concetti di Digital Twin, Digital Shadow e Digital Thread, collegandoli in modo chiaro all'applicazione che abbiamo sviluppato. Verranno esplorate le definizioni di ciascun concetto, ne saranno delineate le origini e saranno esaminati i vantaggi derivanti dalla loro implementazione. Attraverso questa panoramica teorica, si mira a creare una comprensione completa del contesto in cui si inserisce l'applicazione, evidenziando l'importante ruolo che questi concetti hanno svolto nella progettazione e nello sviluppo della piattaforma.
    
%     \item Tecnologie usate \\
%     Questa sezione si concentra sulle tecnologie essenziali utilizzate per lo sviluppo della piattaforma. Verrà eseguita un'analisi dettagliata dei vari linguaggi di programmazione e degli strumenti web utilizzati, esplorando le loro caratteristiche, le loro applicazioni e i vantaggi derivanti dalla loro adozione. Questo capitolo fornisce una base tecnica, permettendo di comprendere appieno le competenze necessarie alla realizzazione dell'applicazione web.
    
%     \item Visualizzazione dei Percorsi Escursionistici \\
%     In questa sezione verranno presentate le numerose schermate e le funzionalità dell'applicazione web sviluppata. Oltre a fornire una panoramica delle schermate dell'applicazione, verranno esplorate anche le sfide tecniche e progettuali affrontate durante la fase del processo di sviluppo. Attraverso esempi di codice e spiegazioni dettagliate, verranno mostrate come sono stati affrontati problemi specifici. 

% \end{enumerate}

% In sintesi, questa tesi si propone di migliorare l'esperienza degli escursionisti nella regione Emilia-Romagna, interagendo e facendosi aiutare anche dagli utenti, sfruttando appieno le opportunità dell'era dell'Industria 4.0.


\addcontentsline{toc}{chapter}{Introduzione}
Nell’ultimo periodo, con l’arrivo dell’Industria 4.0, abbiamo assistito ad un’accelerazione straordinaria nell’adozione di tecnologie avanzate rivoluzionando molti settori, tra cui il turismo~\cite{Turismo} e le attività escursionistiche, portando inoltre all’utilizzo di molte tecnologie innovative, tra cui il Digital Twin, il Digital Shadow e il Digital Thread.

Il contesto in cui si colloca questa tesi è proprio questo scenario di trasformazione. In particolare, questa tesi si propone di presentare una piattaforma web progettata per offrire un servizio dedicato agli appassionati delle escursioni nella regione Emilia-Romagna. L'obiettivo primario di questa piattaforma è quello di arricchire l'esperienza degli escursionisti, integrando i principali concetti di Digital Twin, Digital Shadow e Digital Thread.

La piattaforma permette agli escursionisti non solo di esplorare e pianificare i loro percorsi rendendo questi disponibili sotto forma di mappa interattiva, ma offre anche la possibilità di contribuire attivamente all'arricchimento dei dati. Gli utenti hanno la possibilità di condividere informazioni sui singoli punti di interesse lungo i percorsi, come ad esempio gli orari di apertura dei musei o dettagliate descrizioni di siti storici, migliorando così l'esperienza per tutti gli utenti. Nel frattempo, gli amministratori della piattaforma mantengono il controllo dei dati inseriti, garantendo l'accuratezza e la qualità delle informazioni condivise.

Per quanto riguarda la struttura della tesi, il documento è suddiviso in tre sezioni principali:
\newpage
\begin{enumerate}
    \item Contesto \\
    Questa sezione offre un'analisi approfondita dei concetti di Digital Twin, Digital Shadow e Digital Thread, stabilendo chiaramente i legami con l'applicazione sviluppata. Verranno esplorate in dettaglio le definizioni di ciascun concetto, ne saranno tracciate le origini storiche e saranno esaminati i vantaggi legati alla loro implementazione. Attraverso questa esaustiva panoramica teorica, l'obiettivo è quello di creare una comprensione completa del contesto in cui si inserisce l'applicazione, mettendo in luce l'importante ruolo che questi concetti hanno svolto nella progettazione e nello sviluppo della piattaforma.
    
    \item Tecnologie Usate \\
    Questa sezione si concentra sulle tecnologie fondamentali utilizzate per la realizzazione della piattaforma. Verrà eseguita un'analisi approfondita dei vari linguaggi di programmazione e degli strumenti web impiegati, esplorandone le caratteristiche, le applicazioni e i benefici derivanti dalla loro adozione. Questo capitolo offre una solida base tecnica, consentendo ai lettori di acquisire una comprensione completa delle competenze necessarie per lo sviluppo della piattaforma.
    
    \item Visualizzazione dei Percorsi Escursionistici \\
    In questa sezione verranno presentate dettagliatamente le numerose schermate e le funzionalità dell'applicazione web sviluppata.
    Oltre a fornire una panoramica esaustiva delle schermate dell'applicazione, verranno esplorate anche le sfide tecniche e progettuali affrontate in ogni fase del processo di sviluppo. Attraverso esempi di codice e spiegazioni dettagliate, sono stati affrontati problemi specifici e ottimizzato l'interazione tra utente e piattaforma.
    Questa sezione serve per offrire una panoramica approfondita sulle decisioni di design, le scelte tecnologiche e le soluzioni creative che hanno contribuito a creare questa piattaforma.
\end{enumerate}

In sintesi, questa tesi si pone obiettivo di migliorare l'esperienza degli escursionisti nella regione Emilia-Romagna, coinvolgendo attivamente gli utenti stessi e capitalizzando appieno le opportunità offerte dall'era dell'Industria 4.0.


% In sintesi, questa tesi si propone di migliorare l'esperienza degli escursionisti nella regione Emilia-Romagna, interagendo e facendosi aiutare anche dagli utenti, sfruttando le opportunità dell'era dell'Industria 4.0.

% \addcontentsline{toc}{chapter}{Introduzione}
% Negli ultimi anni, l'adozione accelerata delle tecnologie avanzate nell'ambito dell'Industria 4.0 ha portato ad una rivoluzione in diversi settori. Tra questi settori che hanno sperimentato una trasformazione significativa rientrano il turismo~\cite{Turismo} e le attività escursionistiche, che hanno visto l'introduzione di tecnologie innovative come il Digital Twin, il Digital Shadow e il Digital Thread.

% Il contesto in cui si colloca questa tesi è proprio questo scenario di trasformazione. In particolare, questa tesi propone una piattaforma web progettata per offrire un servizio agli amanti delle escursioni nella regione Emilia-Romagna. L'obiettivo principale di questa piattaforma è migliorare l'esperienza degli escursionisti, integrando in modo intelligente i concetti chiave di Digital Twin, Digital Shadow e Digital Thread.

% L'obbiettivo di questa piattaforma è migliorare in modo significativo l'esperienza degli escursionisti, sfruttando in maniera intelligente i concetti chiave di Digital Twin, Digital Shadow e Digital Thread.

% La piattaforma non si limita a consentire agli appassionati di esplorare e pianificare i loro percorsi, ma offre anche la possibilità di contribuire in modo attivo e collaborativo. Gli utenti hanno la facoltà di arricchire la piattaforma aggiungendo informazioni dettagliate sui punti di interesse lungo i percorsi. Questo potrebbe includere orari di apertura di musei o descrizioni approfondite di siti storici. Questo approccio contribuisce non solo a migliorare l'esperienza di ciascun utente, ma anche a creare una comunità di escursionisti che condivide conoscenze e informazioni preziose.

% Gli amministratori della piattaforma hanno il compito di garantire l'accuratezza e la qualità delle informazioni condivise, mantenendo al contempo un livello di controllo sui dati inseriti.

% Riguardo alla struttura di questa tesi, il documento è organizzato in tre sezioni principali:
% \begin{enumerate}
%     \item Contesto \\
%     Questa sezione offre un'analisi approfondita dei concetti di Digital Twin, Digital Shadow e Digital Thread, collegandoli in modo chiaro all'applicazione che abbiamo sviluppato. Verranno esplorate le definizioni di ciascun concetto, ne saranno delineate le origini e saranno esaminati i vantaggi derivanti dalla loro implementazione. Attraverso questa panoramica teorica, si mira a creare una comprensione completa del contesto in cui si inserisce l'applicazione, evidenziando l'importante ruolo che questi concetti hanno svolto nella progettazione e nello sviluppo della piattaforma.
    
%     \item Tecnologie usate \\
%     Questa sezione si concentra sulle tecnologie essenziali utilizzate per lo sviluppo della piattaforma. Verrà eseguita un'analisi dettagliata dei vari linguaggi di programmazione e degli strumenti web utilizzati, esplorando le loro caratteristiche, le loro applicazioni e i vantaggi derivanti dalla loro adozione. Questo capitolo fornisce una base tecnica, permettendo di comprendere appieno le competenze necessarie alla realizzazione dell'applicazione web.
    
%     \item Visualizzazione dei Percorsi Escursionistici \\
%     Questa sezione presenterà varie schermate e funzionalità dell'applicazione web sviluppata. Inoltre, verranno esposte le sfide tecniche e progettuali affrontate durante tutto il processo di sviluppo.
% \end{enumerate}

% In sintesi, questa tesi si propone di migliorare l'esperienza degli escursionisti nella regione Emilia-Romagna, interagendo e facendosi aiutare anche dagli utenti, sfruttando appieno le opportunità dell'era dell'Industria 4.0.

% \addcontentsline{toc}{chapter}{Introduzione}
% Nell'ultimo periodo, con l'arrivo dell’Industria 4.0, abbiamo assistito ad un'accelerazione straordinaria nell'adozione di tecnologie avanzate rivoluzionando molti settori, tra cui il turismo~\cite{Turismo} e le attività escursionistiche, portando inoltre all'utilizzo di molte tecnologie innovative, tra cui il Digital Twin, il Digital Shadow e il Digital Thread.

% La presente tesi si inserisce in questo contesto, proponendo una piattaforma web progettata per offrire un servizio agli amanti delle escursioni nella regione Emilia-Romagna. Questa piattaforma punta a migliorare l'esperienza degli escursionisti integrando i concetti chiave di Digital Twin, Digital Shadow e Digital Thread.


% % \begin{itemize}
% %     \item Digital Twin è un concetto avanzato che mira a creare una rappresentazione digitale precisa di oggetti fisici o processi del mondo reale. Nel contesto della nostra applicazione, il Digital Twin è rappresentato dai sentieri escursionistici della regione Emilia-Romagna, che sono accuratamente mappati e descritti all'interno della piattaforma. Questo permette agli utenti di esplorare virtualmente i percorsi, ottenendo una visione dettagliata delle loro caratteristiche e dei vari punti di interesse.

% %     \item Digital Shadow si concentra sulla raccolta e l'analisi dei dati generati dalle interazioni digitali. Nel caso della nostra applicazione, il Digital Shadow si traduce nella raccolta delle preferenze degli utenti e delle interazioni con la piattaforma stessa.

% %     \item Digital Thread è un concetto complementare che collega i dati e le informazioni lungo l'intero ciclo di vita di un prodotto o di un processo. Nel contesto della nostra applicazione, il Digital Thread rappresenta la connessione tra i dati dei percorsi escursionistici, le preferenze degli utenti e le informazioni sui punti di interesse. Questo collegamento permette una gestione più efficiente delle informazioni e una migliore personalizzazione delle esperienze degli utenti.
% % \end{itemize}

% La piattaforma permette agli escursionisti non solo di esplorare e pianificare i loro percorsi rendendo questi disponibili sotto forma di mappa, ma anche di contribuire attivamente aggiungendo informazioni utili sui singoli punti di interesse (come per esempio l'orario di apertura di un museo), migliorando così l'esperienza di tutti gli utenti, permettendo comunque agli amministratori della piattaforma un certo livello di controllo sui dati inseriti.

% Per quanto riguarda la struttura della tesi, il documento sarà suddiviso in tre sezioni principali. 

% \begin{enumerate}
%     \item Nel primo capitolo verrà eseguita un'analisi approfondita dei concetti di Digital Twin, Digital Shadow e Digital Thread, mettendoli in relazione con l'applicazione che abbiamo sviluppato. Esploreremo le definizioni, le origini e i vantaggi di ciascun concetto. Sarà fornita un'ampia panoramica teorica per comprendere il contesto in cui l'applicazione si inserisce, mettendo in luce il ruolo cruciale di questi concetti nella progettazione e nello sviluppo dell'applicazione.

%     \item  Nel secondo capitolo, verranno esaminate le tecnologie fondamentali utilizzate per sviluppare la piattaforma. Verrà eseguita un'analisi dettagliata dei vari linguaggi e strumenti di programmazione web impiegati, esplorando le loro caratteristiche, applicazioni e vantaggi. Questo capitolo ci permetterà di comprendere le basi tecniche che hanno reso possibile la realizzazione della nostra applicazione.
    
%     \item Nel terzo capitolo verranno mostrate varie schermate dell'applicazione web realizzata, esplorando inoltre le sfide tecniche e progettuali affrontate durante lo sviluppo, illustrando le varie schermate e funzionalità dell'applicazione web.
% \end{enumerate}

% In sintesi, questa tesi si prefigge di dimostrare come l'integrazione dei concetti di Digital Twin, Digital Shadow e Digital Thread possa migliorare in modo significativo l'esperienza degli escursionisti, offrendo un servizio avanzato e personalizzato nella regione Emilia-Romagna attraverso l'uso innovativo delle tecnologie web.

%%%%%%%%%%%%%%%%%%%%%%%%%%%%%%%%%%%%%%%%%non numera l'ultima pagina sinistra
\clearpage{\pagestyle{empty}\cleardoublepage}