%%%%%%%%%%%%%%%%%%%%%%%%%%%%%%%%%%%%%%%%%non numera l'ultima pagina sinistra
\clearpage{\pagestyle{empty}\cleardoublepage}
%%%%%%%%%%%%%%%%%%%%%%%%%%%%%%%%%%%%%%%%%per fare le conclusioni
\chapter*{Conclusioni}
%%%%%%%%%%%%%%%%%%%%%%%%%%%%%%%%%%%%%%%%%imposta l'intestazione di pagina
\rhead[\fancyplain{}{\bfseries
CONCLUSIONI}]{\fancyplain{}{\bfseries\thepage}}
\lhead[\fancyplain{}{\bfseries\thepage}]{\fancyplain{}{\bfseries
CONCLUSIONI}}
%%%%%%%%%%%%%%%%%%%%%%%%%%%%%%%%%%%%%%%%%aggiunge la voce Conclusioni
                                        %   nell'indice
\addcontentsline{toc}{chapter}{Conclusioni} 
L'obiettivo di questo progetto di tesi è la realizzazione di un'applicazione web dedicata agli escursionisti nella regione Emilia-Romagna.

Sono stati esaminati i concetti di Digital Twin, Digital Shadow e Digital Thread, analizzandone prevalentemente le definizioni, l'origine e i vantaggi. Si è compreso il ruolo fondamentale del Digital Twin nell'Industria 4.0, del Digital Shadow nella raccolta dei dati degli utenti e del Digital Thread nella gestione continua delle informazioni.
Sono state analizzate le tecnologie fondamentali per lo sviluppo di questa piattaforma tra cui HTML, CSS, JavaScript e altre librerie e framework pertinenti. 
Infine sono state discusse le sfide tecniche affrontate durante lo sviluppo, focalizzandosi sulla visualizzazione dei percorsi escursionistici e sulla gestione dei dati. Sono state presentate le varie funzionalità dell'applicazione, tra cui la visualizzazione dei percorsi, i servizi di geo-localizzazione, la registrazione degli utenti e la possibilità di arricchire i dati relativi ai punti di interesse.

È importante notare che, mediante la modifica dei dati forniti per la creazione del database, è possibile estendere l'applicazione per consentire la visualizzazione di percorsi escursionistici in qualsiasi parte del mondo. Inoltre, siccome è già stata implementata la funzionalità di geo-localizzazione, potrebbe essere esplorata la possibilità di tracciare gli utenti lungo i percorsi e, potenzialmente, rendere possibile un'interazione tra gli escursionisti lungo lo stesso sentiero. Questi sviluppi futuri potrebbero ulteriormente trasformare l'applicazione in un social network dedicato agli amanti delle escursioni. In futuro potrebbero anche essere aggiunte integrazioni aggiuntive, usando i servizi offerti da \href{https://www.tripadvisor.com/}{\textit{Tripadvisor}}, da \href{https://www.google.com/maps/d/viewer?msa=0&iwloc=00044592e21a9f5590527&ved=0COIBEJwFSAE&sa=X&ei=I0ShTO3lOqiijQOHmtz6Aw&mid=1eLqvkQ9wGvMRVrAQsm5g7EdlnSY&ll=39.04394865349766,-76.85871&z=10}{\textit{Google places}} o da \href{https://www.arpae.it/it}{\textit{Arpae}}. Infine potrebbe essere aggiunta la possibilità di mostrare il meteo sulla regione o di mostrare le immagini satellitari riguardanti i punti di interesse.

\clearpage{\pagestyle{empty}\cleardoublepage}